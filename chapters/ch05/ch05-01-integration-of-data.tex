\section{데이터의 통합}

마케팅 테크놀로지(Marketing Technology)의 눈부신 발전으로 수많은 데이터가 쏟아지고 있지만, 아이러니하게도 우리는 고객을 더 모르게 되었다\sidenote{가트너(Gartner)의 \enquote{2020년 마케팅 데이터 및 분석 설문조사}에 따르면, 데이터의 양은 폭발적으로 증가했지만 마케팅 분석이 실제 의사결정에 영향을 미치는 비율은 53\%에 불과했다. 이는 데이터의 풍요가 통찰의 빈곤(Insight Gap)으로 이어지는 역설을 보여준다.}\cite{gartner2020analytics}. 데이터는 넘쳐나는데 정작 고객의 마음은 흩어져 있어 보이지 않기 때문이다. 이것은 데이터의 양(Volume) 문제가 아니라, 데이터가 담고 있는 \texttt{의미(Meaning)}가 단절되어 있기 때문에 발생하는 문제다.

\vspace{1em}
\noindent\textbf{파편화된 진실: 통계의 한계}

오늘날 마케팅 현장에서 고객은 수많은 ID로 파편화되어 존재한다. 쿠키 ID, ADID, 회원 번호, CRM 식별자 등 수십 개의 꼬리표가 붙지만, 정작 이 모든 것이 \enquote{한 사람}의 이야기라는 사실은 자주 망각된다.

우리는 흔히 \enquote{데이터 통합}을 이야기하며 여러 채널의 데이터를 한곳에 모으는 물리적인 결합을 시도한다\sidenote{데이터 웨어하우스(Data Warehouse)나 데이터 레이크(Data Lake)에 모든 소스의 데이터를 쏟아붓는 것은 진정한 의미의 통합(Integration)이라기보다 보관(Archiving)에 가깝다. 물리적으로 한 공간에 있다고 해서, 서로 다른 맥락의 데이터가 논리적으로 연결되는 것은 아니기 때문이다.}. 하지만 이것만으로는 충분하지 않다. 
단순히 웹사이트 방문 기록과 오프라인 구매 내역을 엑셀로 합친다고 해서 고객의 의도가 읽히지는 않는다. 
기존의 통계적 방식이나 머신러닝(Machine Learning) 기법은 \enquote{이 상품을 산 사람이 저 상품도 살 확률이 높다}는 상관관계(Correlation)는 기가 막히게 찾아 내지만, 
\enquote{왜(Why)} 샀는지에 대한 인과관계(Causality)는 설명하지 못한다.

숫자는 거짓말을 하지 않는 것은 분명하다. 하지만 진실의 일부만을 말해줄 뿐이다. 
또 하나 우리가 생각해야할 것은, 고객은 확률적 분포가 아니라, 뚜렷한 목적과 맥락을 가진 주체라는 것이다. 
따라서 진정한 의미의 데이터 통합은 물리적 파이프라인의 연결이 아니라, 고객이 남긴 파편화된 흔적들을 하나의 \texttt{맥락(Context)}으로 꿰어내는 작업이어야 한다.

\vspace{1em}
\noindent\textbf{의미를 이해하는 데이터: 심볼릭 AI의 필요성}

여기서 우리는 \index{심볼릭 AI}\texttt{심볼릭 AI(Symbolic AI)}\sidenote{심볼릭 AI는 데이터를 숫자나 확률이 아닌, 인간이 이해하는 \enquote{개념(기호)}과 \enquote{논리}로 다루는 인공지능이다. 이산수학(Discrete Mathematics)과 논리학(Logic)을 이론적 배경으로 하며, \enquote{사과는 과일이다}처럼 명시적인 지식을 기반으로 추론하므로 결과의 이유를 설명할 수 있다.}의 가치에 주목해야 한다. 
예를 들어보자. 어떤 고객이 고객센터에 불만 전화를 건 뒤, 앱에서 탈퇴 버튼을 눌렀다고 가정해 보자.
\begin{itemize}
    \item \textbf{통계적 접근}: \enquote{최근 통화 이력이 있는 고객의 이탈 확률 85\%.} (현상 예측)
    \item \textbf{심볼릭 AI 활용 접근}: \enquote{불만(Negative Experience) 사건 발생 $\rightarrow$ 신뢰(Trust) 손상 $\rightarrow$ 관계 단절(Churn) 의도 실행.} (원인 이해)
\end{itemize}

통계적 AI가 \enquote{무슨 일이 일어날까?}를 맞히는 데 탁월하다면, 심볼릭 AI는 \enquote{왜 그런 일이 일어났는가?}를 이해하고 설명하는 데 강점이 있다. 마케팅은 궁극적으로 사람의 마음을 얻는 일이다. 마음을 얻기 위해서는 확률 계산보다 \enquote{맥락에 대한 이해}가 선행되어야 한다. 우리가 앞서 설계한 온톨로지 모델(욕구의 위계, 인과성, 시간의 축적)은 바로 이 심볼릭 AI가 데이터를 이해하고 해석할 수 있도록 돕는 지식의 지도(Knowledge Graph) 역할을 한다.

\vspace{1em}
\noindent\textbf{온톨로지: 연결의 매개체}

결국 온톨로지를 기반으로 데이터를 처리한다는 것은, 기계에게 우리 브랜드의 \texttt{언어}를 가르치는 것과 같다. 기계가 단순히 \enquote{클릭 수}를 세는 것을 넘어, 그 클릭이 \enquote{탐색 욕구}의 표현인지, \enquote{구매 의도}의 발현인지를 우리의 마케팅 문법으로 해석하게 만드는 것이다.

이 책에서 정의하는 \enquote{일관된 경험의 구현}은 바로 이 지점에서 완성된다. 모든 채널의 데이터가 하나의 온톨로지 구조 위에서 통합될 때, 시스템은 비로소 고객을 \enquote{식별 번호}가 아닌 \enquote{어제 만났던 바로 그 사람}으로 기억하게 된다. 기술적으로 가장 고도화된 시스템이 역설적으로 가장 인간적인 대우를 가능하게 만드는 것이다. 이것이 곧 마케팅 온톨로지가 지향하는 \texttt{관계의 기술(Technology of Relationship)}이다.
