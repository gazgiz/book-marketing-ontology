\chapter*{서문}
\addcontentsline{toc}{chapter}{서문} % Add the preface to the table of contents as a chapter
\setstretch{1.6}
마케팅과 온톨로지는 모두 {\normalfont\texttt{관계}}라는 단어를 사용한다. 그러나 같은 단어를 사용한다고 해서 이들이 같은 의미를 담고 있는 것은 아니다. 두 영역에서 관계라는 단어는 서로 완전히 다른 개념적
틀을 가지고 사용된다. 이 점을 서문의 시작에서 분명하게 하고자 한다.

먼저 \texttt{마케팅에서의 관계(relationship)}는 주로 사람과 사람, 혹은 고객과 브랜드 사이의 심리적이고 정서적인 유대감을 의미한다. 이 관계는 신뢰, 충성도, 애착, 만족감과 같은 감성적이며 주
관적인 연결성을 나타낸다. 마케팅은 바로 이 유대감을 형성하고 유지하며 강화하는 활동으로, 최근의 마케팅 연구들은 이를 \enquote{관계의 설계}\parencite{gummesson2008}로 표현하기도 한다.
반면, \texttt{온톨로지에서의 관계(relation)}는 주관적이거나 감성적이지 않다. 온톨로지에서의 관계는 명확히 정의된 \texttt{엔티티(entity)}들 사이의 구조적이고 객관적인 연결이다. 특정 \texttt{주체(subject)}와 \texttt{객체(object)}가 어떤 \texttt{의미적 관계(predicate)}로 연결되어 있는지 논리적으로 명시하는 것이 온톨로지적 관계이다.
예를 들어,\enquote{고객이 제품을 구매하다}, 혹은 \enquote{사용자가 서비스를 이용하다}와 같은 관계가 온톨로지적 의미의 관계이다.

이 책은 마케팅을 온톨로지적 방법으로 분석하고 활용하는 것을 목적으로 한다. 그렇기에 두 영역에서 사용되는 용어와 개념을 혼동하지 않도록 초반부터 명확히 구분하는 것이 중요하다. 마케팅적 관계
의 감성적이고 심리적인 요소는, 온톨로지에서 사용되는 객관적이고 구조적인 관계로 명료하게 재구성되어야 한다. 이를 통해 비로소 마케팅을 온톨로지적으로 명확하게 분석하고, 설계하고, 실행할 수 있다.

\begin{flushright}
	\textit{저자}
\end{flushright}

\setstretch{1.2}