\section{관계의 목적}
앞에서 우리는 고객, 브랜드, 제품, 채널, 시간, 그리고 고객의 내면 요소까지 포함한 관계의 구조를 살펴보았다. 그러나 관계의 언어를 아무리 정교하게 설명하더라도, 한 가지 질문이 남는다. 이 모든 관계를 왜 설계하려 하는가, 그리고 그 관계는 무엇을 위해 존재해야 하는가 하는 문제다. 관계의 목적이 불명확한 상태에서 이루어지는 마케팅 활동은, 결국 더 많은 메시지를 더 자주 보내자는 수준의 결론으로 흘러가기 쉽다. 메시지 과잉의 시대에는 이것이 곧 노이즈의 생산을 의미한다.

관계의 목적을 이야기할 때, 가장 먼저 사라지기 쉬운 것이 ‘제품’이다. 고객과 브랜드의 심리적 유대만 강조하다 보면, 정작 무엇을 매개로 관계가 형성되고 유지되는지에 대한 논의가 뒤로 밀린다.
하지만 고객은 브랜드 자체를 소비하지 않는다. 고객은 자신의 문제 상황을 해결해 줄 수단으로서 제품을 구매하고, 그 제품이 만들어 내는 경험을 통해 브랜드를 판단한다.
즉, 제품은 고객–브랜드 관계가 현실에서 구현되는 매개체이자, 관계의 목적이 구체적인 형태로 드러나는 자리다. 따라서 관계의 목적을 이해하려면, 결국 \enquote{고객이 어떤 제품을 어떻게 선택하도록 돕고 있는가}라는 질문으로 돌아가야 한다.
관계는 단지 \enquote{좋은 사이}를 유지하기 위한 감정적 장치가 아니다. 마케팅 관점에서 관계의 가장 근본적인 목적은 바로 이 선택 과정에서의 판단 비용을 줄이는 데 있다.
고객이 매번 새로운 선택을 할 때마다 모든 제품을 처음부터 비교·분석할 수는 없다. 이때 고객은 브랜드와 관계뿐 아니라, 특정 제품군과의 누적된 경험을 함께 사용한다. 
\enquote{이 브랜드의 이 라인 제품은 대략 이런 품질과 이런 느낌을 줄 것이다}라는 기대가 형성되어 있을 때, 고객은 복잡한 계산 없이도 특정 브랜드–제품 묶음을 빠르게 선택할 수 있다. 
관계는 이렇게 고객의 머릿속에 \enquote{브랜드–제품 패턴}을 만들어 놓고, 그 패턴을 통해 판단의 지름길(\index{휴리스틱}휴리스틱\sidenote{휴리스틱은 심리학,인지과학에서 자주쓰이는 말로 모든 정보를 완벽하게 따져보지 못할때, 경험과 직관을 활용해 \enquote{그럴듯한} 결론에 빠르게 도달하게 해주는 판단의 요령을 뜻한다.})을 제공한다.

브랜드 입장에서 보았을 때도 마찬가지다. 브랜드는 시장에 존재하는 모든 잠재 고객을 동일한 강도로 설득할 수 없다. 누구에게, 어떤 강도로 다가갈 것인지 선택해야 한다. 이때 관계는 한정된 자원을 어디에 우선 배분할 것인지 결정하게 해 주는 기준이 된다. 이미 일정 수준의 신뢰와 상호 이해가 형성된 고객은 그렇지 않은 고객에 비해 더 낮은 설득 비용으로 더 높은 반응을 기대할 수 있는 집단이 된다. 관계의 목적은 따라서, 고객과 브랜드 양쪽 모두의 판단 비용을 동시에 낮추는 데 있다.
