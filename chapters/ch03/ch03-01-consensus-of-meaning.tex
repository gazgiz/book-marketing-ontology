\section{의미의 합의}

우리가 사용하는 언어는 근본적으로 불완전하다. 같은 단어를 사용한다고 해서 같은 대상을 지칭하는 것은 아니다. 2장에서 살펴보았듯이, \enquote{충성도}나 \enquote{가치} 같은 핵심적인 비즈니스 용어조차 개인의 경험과 직관에 따라 전혀 다른 의미로 해석되곤 한다.\sidenote{이러한 언어적 모호성은 소통의 비용을 증가시키고, 판단의 일관성을 무너뜨리는 주된 원인이 된다.}

\index{온톨로지}\texttt{온톨로지(Ontology)}의 본질은 이러한 모호함을 걷어내기 위한 \enquote{의미의 합의} 과정에 있다. 기술적인 정의는 잠시 배제하고 본질적인 관점에서 설명하자면, 온톨로지는 \enquote{우리가 무엇에 대해 이야기하고 있는가}를 명확히 정의하고, 그 정의를 지키기로 한 구성원\sidenote{여기서 구성원이란 사람뿐만 아니라, 시스템을 구성하는 모든 개체를 의미한다. 즉, 데이터를 생성하고 소비하는 소프트웨어, 그리고 이를 학습하는 인공지능(AI)까지를 포함하는 포괄적인 개념이다.} 간의 \texttt{약속(Promise)}이다.

\vspace{1em}
\noindent\textbf{언어의 불확실성 (Uncertainty of Language)}

일상 대화에서 약간의 오해는 유연하게 넘어갈 수 있다. 하지만 비즈니스, 특히 데이터와 시스템이 개입되는 환경에서 모호한 정의는 치명적인 결과를 초래한다.

예를 들어, A라는 시스템이 정의하는 \enquote{신규 고객}은 \enquote{이번 달에 처음으로 회원 가입한 사람}일 수 있다. 반면, B라는 시스템이 처리하는 \enquote{신규 고객}은 \enquote{이번 달에 첫 구매를 발생시킨 사람}일 수 있다. 이 두 정의가 혼재된 상태에서 \enquote{신규 고객 유치 전략}을 데이터를 통해 분석하려 한다면, 결과는 왜곡될 수밖에 없다. 서로 다른 대상을 같은 이름으로 부르고 있기 때문이다.

이러한 불확실성은 단순히 용어집(Glossary)\sidenote{용어집이란 특정 도메인에서 사용되는 용어와 그 정의를 모아둔 문서를 말한다. 이는 구성원들이 동일한 단어를 동일한 의미로 이해하도록 돕는 1차적인 기준점 역할을 수행하여 의사소통의 혼선을 줄이는 데 기여한다.}을 만든다고 해결되지 않는다. 용어집은 단어의 뜻을 나열할 뿐, 그 단어들이 서로 어떤 관계를 맺고 있는지, 어떤 맥락에서 유효한지까지는 강제하지 못하기 때문이다. 따라서 우리에게는 단순한 사전적 정의를 넘어, 대상을 바라보는 관점을 일치시키는 더 강력한 도구가 필요하다.

\vspace{1em}
\noindent\textbf{존재에 대한 규정 (Defining Existence)}

온톨로지라는 단어는 원래 \enquote{존재론}을 뜻하는 철학 용어에서 유래했다. 즉, \enquote{무엇이 존재하는가}를 규명하는 학문이다. 마케팅 시스템을 구축하는 과정에서도 이 질문은 가장 먼저 던져져야 한다.

\textit{우리 비즈니스 세계에 \texttt{무엇이 존재하는가?}}

이 질문에 답하는 것이 온톨로지 설계의 시작이다. 우리는 \enquote{고객}이 존재한다고 합의한다. \enquote{상품}이 존재하고, \enquote{구매}라는 행위가 존재한다고 합의한다. 너무나 당연해 보이는 이 과정이 중요한 이유는, 우리가 합의하여 정의하지 않은 대상은 시스템 안에서 존재할 수 없기 때문이다.

명확하게 정의된 존재만이 데이터로 포착될 수 있고, 논리의 재료로 사용될 수 있다. 온톨로지는 이처럼 모호한 현실 세계의 대상들을 명확한 경계를 가진 \enquote{개념}으로 확정 짓는 작업이다.

\vspace{1em}
\noindent\textbf{약속으로서의 구조 (Structure as a Promise)}

이렇게 합의된 정의는 단순한 텍스트로 남아서는 안 된다. 그것은 시스템 전체가 반드시 지켜야 할 \texttt{약속}이 되어야 한다.

\textit{우리가 \texttt{구매}라고 말할 때는, 반드시 \texttt{결제 완료} 상태이면서 \texttt{환불되지 않은} 거래만을 의미하기로 약속한다.}

이 약속이 지켜질 때, 비로소 데이터는 신뢰를 얻는다. 온톨로지는 이 약속을 시스템의 언어, 즉 구조적인 형태로 기록한 것이다. 누군가의 머릿속에만 있는 암묵적인 규칙이 아니라, 시스템이 검증하고 강제할 수 있는 형태로 명문화된 약속이다.

결국 온톨로지를 구축한다는 것은, 내재화된 직관적인 언어를 걷어내고, 모두가 동의하고 검증할 수 있는 \enquote{약속의 언어}를 만드는 과정이다. 이 약속이 견고할수록, 우리는 불필요한 해석의 논쟁 없이 데이터가 가리키는 현상 그 자체에 집중할 수 있게 된다.
