\section{관계의 구조화}

의미의 모호함을 걷어내기 위해 \enquote{약속}이 필요하다는 점은 자명하다. 그렇다면 이 약속을 어떻게 시스템에 기록하고 강제할 수 있을까? 

단순한 단어의 정의만으로는 부족하다. 약속을 구체적이고 명확하게 표현하기 위해서는, 단어와 단어 사이의 관계를 규정하는 \enquote{문법(Grammar)}이 필요하다. 우리가 정의한 개념들이 비즈니스의 \enquote{어휘(Vocabulary)}라면, 온톨로지는 이 어휘들을 엮어 흔들리지 않는 약속의 문장을 만드는 \enquote{문법}이다.

\vspace{1em}
\noindent\textbf{단어에서 문장으로 (From Words to Sentences)}

\enquote{쿠키(Cookie)}라는 단어를 생각해보자. 이 단어 하나만으로는 그것이 \enquote{먹는 과자}인지, \enquote{웹 브라우저의 데이터}인지 알 수 없다. 맥락이 부재하기 때문이다. 하지만 다음과 같이 문장으로 표현하면 의미는 명확해진다.

\begin{itemize}
    \item \texttt{Cookie}는 \texttt{사용자 정보}를 \texttt{저장한다(Stores)}.
    \item \texttt{Cookie}는 \texttt{초콜릿}을 \texttt{함유한다(Contains)}.
\end{itemize}

첫 번째 문장에서의 Cookie는 IT 기술 용어이고, 두 번째 문장에서는 간식이다. 이처럼 단어의 진짜 의미는 독립적으로 존재하지 않으며, 다른 단어와의 \texttt{관계} 속에서 비로소 결정된다. 온톨로지는 바로 이 \enquote{관계}를 구조적으로 정의하는 체계다.

\vspace{1em}
\noindent\textbf{의미의 최소 단위: 주어-서술어-목적어}

온톨로지에서 모든 정보는 가장 단순하고 명확한 구조인 \enquote{주어(Subject) - 서술어(Predicate) - 목적어(Object)}의 형태로 기록된다. 이 세 가지 요소가 모여 하나의 단위를 구성하기 때문에, 이를 \texttt{트리플(Triple)}이라 부른다.

\begin{itemize}
    \item \textbf{주어(Subject)}: 행위의 주체 또는 설명의 대상 (예: 고객)
    \item \textbf{서술어(Predicate)}: 주어와 목적어 사이의 관계 (예: ~을 구매하다)
    \item \textbf{목적어(Object)}: 행위의 대상 또는 속성값 (예: 상품)
\end{itemize}

우리가 흔히 사용하는 \enquote{고객이 상품을 구매했다}라는 문장도 이 구조를 따른다.

\begin{center}
    \texttt{고객(Customer)} $\xrightarrow{\text{구매하다(Buys)}}$ \texttt{상품(Product)}
\end{center}

이처럼 세상의 모든 복잡한 현상을 이 세 가지 요소를 사용해 \enquote{누가(Who), 무엇을(What), 어떻게(How)} 연결되는지로 분해하여 기록하는 것이 온톨로지 모델링의 핵심이다.

\begin{figure}[ht]
    \centering
    \includegraphics[width=0.9\textwidth]{figures/ch03/ontology_graph_visualization.png}
    \caption[복잡하게 연결된 온톨로지 그래프]{복잡하게 연결된 온톨로지 그래프의 예시. 단어(Node)와 관계(Edge)가 거미줄처럼 연결되어 거대한 의미의 그물망을 형성한다.}
    \label{fig:ontology-graph}
\end{figure}

\vspace{1em}
\noindent\textbf{구조화의 힘 (Power of Structure)}

왜 굳이 이렇게 딱딱한 구조로 정보를 기록해야 할까? 그 이유는 \texttt{기계가 이해할 수 있는 형태}이기 때문이다.

사람은 \enquote{김철수 고객님이 어제 A 상품을 샀어요}라는 문장을 읽고 바로 이해할 수 있다. 하지만 컴퓨터 시스템에게 이 문장은 그저 텍스트 데이터일 뿐이다. 반면, 이를 \texttt{김철수(Subject)} - \texttt{구매하다(Predicate)} - \texttt{A상품(Object)}이라는 구조로 전달하면, 시스템은 비로소 \enquote{구매}라는 관계를 인식하고 처리할 수 있게 된다.

또한 이 구조는 확장이 용이하다. \texttt{A상품}이 어떤 \texttt{카테고리}에 속하는지, 그 카테고리는 어떤 \texttt{상위 분류}에 속하는지를 계속해서 꼬리에 꼬리를 물고 연결할 수 있다.

\begin{center}
    \texttt{고객} $\to$ \texttt{구매} $\to$ \texttt{상품} $\to$ \texttt{소속} $\to$ \texttt{카테고리}
\end{center}

이렇게 연결된 의미의 그물망을 따라가면, \enquote{특정 카테고리를 선호하는 고객}을 찾아내거나, \enquote{함께 구매될 가능성이 높은 상품}을 추천하는 등의 고차원적인 추론이 가능해진다. 이것이 바로 온톨로지가 단순한 데이터 저장소를 넘어, \enquote{지능형 시스템}의 기반이 되는 이유다.

