\section{구조적 해결책}

파편화된 맥락을 다시 연결하고 무질서해지는 논리를 바로잡는 것은 개인의 의지나 노력만으로는 불가능하다. 엔트로피가 증가하는 것은 자연의 섭리이므로, 이를 거스르기 위해서는 그에 상응하는 물리적, 논리적 구속력이 필요하다. 즉, 맥락을 스스로 지탱할 수 있는 견고한 \enquote{구조(Structure)}가 요구된다.

\vspace{1em}
\noindent\textbf{구조로 엔트로피에 맞서다}

구조의 핵심 역할은 정보가 전달되는 경로를 정형화하여 해석의 불확실성을 제거하는 데 있다. 정보가 자연적으로 흩어지려는 성질을 억제하려면, 자의적인 해석이 개입될 여지를 원천적으로 차단해야 한다. 언제 어디서나 동일한 맥락에서 정보가 이해될 수밖에 없도록 만드는 \enquote{논리적 강제성}이야말로 엔트로피에 맞서는 가장 강력한 무기다.

이 구조는 파편화된 문서나 개별적인 기억 속에 존재해서는 안 된다. 언제나 참조 가능하고 검증 가능한 독립적인 실체로 존재해야 한다. 그래야만 시간이 흘러도 그 안에 담긴 의도와 맥락이 변질되지 않고 보전될 수 있다.

\vspace{1em}
\noindent\textbf{판단 기준의 객관화 (Objectification of Standards)}

구조를 세우기 위한 첫 번째 단계는 내재화(Internalized)된 \enquote{판단의 기준}을 밖으로 꺼내어 객관화(Objectification)하는 것이다.

앞서 살펴보았듯이, 직관에 의존한 판단 기준은 검증할 수 없는 블랙박스와 같다. 기준이 내면에 숨겨져 있는 한, 서로 다른 논리적 전제 위에서 평행선을 달리는 논쟁은 멈추지 않는다. 서로 다른 척도를 가지고 대상을 측정하고 있기 때문이다.

해결책은 그 척도 자체를 밖으로 꺼내어 객관적인 실체로 만드는 것이다. \enquote{느낌}이나 \enquote{경험}이 아닌, \enquote{명시된 기준}에 따라 판단이 이루어져야 한다. 판단의 주체를 \enquote{내부의 직관}에서 \enquote{외부의 구조}로 옮기는 것, 그것이 바로 판단 기준의 객관화다.

\vspace{1em}
\noindent\textbf{공용어로서의 논리}

기준이 객관화되고 구조가 갖춰지면, 이제 우리는 모호한 감각이 아닌 \enquote{논리}라는 공용어로 대상을 정의할 수 있게 된다.

추상적인 전략적 의도가 구체적인 실행 과업으로 이어지는 과정이 하나의 투명한 논리 사슬(Logic Chain)로 연결된다. 왜 이 과업을 수행해야 하는지, 이 과업이 상위의 목적과 어떻게 인과적으로 연결되는지(Traceability)가 구조적으로 명확해진다. 이 논리의 사슬이 견고하게 구축될 때, 비로소 맥락은 파편화되지 않고 온전한 의미를 유지할 수 있다.

결국 2장의 제목인 \enquote{판단의 기준}은, 탁월한 직관에 의존하는 것이 아니라, 명확하게 설계되고 합의된 \enquote{논리적 구조} 위에서만 비로소 확립될 수 있다.
