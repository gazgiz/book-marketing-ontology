\section{맥락의 파편화}

설명 가능한 직관은 그 자체로 완결된 논리 구조를 갖는다. 하지만 개별적인 완결성이 전체의 정합성을 보장하지는 않는다. 각자가 독립적으로 수립한 직관의 구조가 서로 호환되지 않을 때, \enquote{맥락의 파편화(Fragmentation of Context)}\sidenote{독립적으로 구조화된 데이터나 정보가 서로 다른 기준과 정의를 따름으로써, 상호 호환되지 않거나 연결될 수 없는 상태.}가 발생한다.

A 마케터는 자신이 수립한 논리에 따라 \enquote{고객 충성도}를 \enquote{구매 빈도}로 정의하고 이를 기준으로 성과를 판단한다. 반면, B 마케터는 동일한 \enquote{고객 충성도}를 \enquote{NPS(순수 추천 지수)}로 정의하고 자신의 논리를 전개한다. 두 사람 모두 자신의 직관을 설명 가능한 형태로 구조화했지만, 이 두 논리는 서로 다른 \enquote{맥락(Context)} 위에 서 있다.

\begin{figure}[h]
    \centering
    \includegraphics[width=0.9\textwidth]{figures/ch02/fragmentation-of-context.png}
    \caption[맥락의 파편화]{맥락의 파편화: 동일한 단어 \enquote{충성도}에 대해 서로 다른 정의를 내림으로써 발생하는 판단의 불일치.}
    \label{fig:fragmentation-of-context}
\end{figure}

이러한 독립된 구조들이 충돌할 때, 우리는 \enquote{판단의 기준}을 상실하게 된다. 각자의 논리 안에서는 모두가 옳지만, 서로의 결론을 비교하거나 검증할 수 있는 공통의 잣대가 없기 때문이다. 설명 가능해진 직관들이 오히려 서로를 반박하는 근거로 사용되며 판단의 혼란을 가중시키는 역설적인 상황이 벌어진다.

\vspace{1em}
\noindent\textbf{합의의 착각 (The Illusion of Agreement)}

파편화가 위험한 진짜 이유는 그것이 겉으로 잘 드러나지 않는다는 데 있다. 우리는 종종 \enquote{같은 단어}를 쓰고 있다는 이유만으로 서로 \enquote{같은 생각}을 하고 있다고 착각한다. 이를 \enquote{비즈니스 동음이의어(Business Homonyms)} 현상이라 부를 수 있다\cite{khostikyan2020significance}.

앞서 예로 든 \enquote{충성도}처럼, \enquote{가치}, \enquote{혁신}, \enquote{인게이지먼트}와 같은 추상적인 마케팅 용어들은 사람마다, 혹은 부서마다 각기 다른 정의와 기대치를 내포하고 있다. 회의실에서는 모두가 고개를 끄덕이며 합의한 것처럼 보이지만, 문을 나서는 순간 각자의 머릿속에 그려진 그림은 전혀 다르다. 이는 겉으로는 소통이 이루어지는 것처럼 보이지만, 실질적인 의미는 전혀 전달되지 않는 \texttt{소통의 단절}\index{소통의 단절}을 낳는다.

\vspace{1em}
\noindent\textbf{맥락의 엔트로피 (Context Entropy)}

설상가상으로, 이러한 맥락은 시간이 지날수록, 그리고 전달 단계가 늘어날수록 자연스럽게 소실된다. 예를 들어, 경영진이 수립한 \enquote{고객 생애 가치(LTV) 증대}라는 전략적 맥락은 본부장에게는 \enquote{재구매율 5\% 상승}으로, 팀장에게는 \enquote{주간 프로모션 문자 발송}으로 번역되어 내려간다. 정작 실무자는 \enquote{금요일 오후 2시 쿠폰 발송}이라는 단순 과업만을 수행하게 된다. 최초의 의도였던 \enquote{고객과의 관계 강화}라는 풍부한 맥락은 사라지고, 기계적인 \enquote{할 일(To-Do)}만 남아 고객에게 피로감을 주는 스팸성 메시지로 전락하는 것이다. 

\begin{marginfigure}
    \centering
    \includegraphics[width=\linewidth]{figures/ch02/context_entropy_vertical.png}
    \caption[맥락의 엔트로피]{맥락의 엔트로피: 전달 단계가 늘어날수록 의도(Strategy)는 희석되고 잡음(Noise)만 남게 된다.}
    \label{fig:context_entropy}
\end{marginfigure}

이처럼 시간이 지날수록 최초의 질서 정연했던 의도가 무너지는 것은 자연스러운 현상이다. 마치 물리학의 엔트로피(무질서도)가 증가하듯, 정보가 전달되는 과정에서의 맥락 또한 의도적인 노력 없이는 필연적으로 파편화되고 무질서해지는 경향이 있기 때문이다. 이를 \enquote{맥락의 엔트로피(Context Entropy)}라 부를 수 있다. 누군가가 끊임없이 맥락을 동기화하고 구조를 유지하려 노력하지 않는다면, 우리의 모든 판단은 결국 각자의 파편적인 논리 속에 갇혀 고립될 수밖에 없다.


