\section{직관의 한계}

마케팅 현장에서 수많은 결정을 내릴 때, 우리는 흔히 \index{감}\enquote{감}\sidenote[][1.5cm]{복잡한 상황에서 즉각적으로 결론에 도달하는 전문가의 직관적 판단 능력.}이나 \index{경험}\enquote{경험}\sidenote[][3.0cm]{과거의 성공과 실패 사례가 뇌에 축적되어 형성된 암묵적 데이터베이스.}에 의존하곤 한다. 이는 비과학적이거나 무리한 방식이 아니라, 오히려 오랜 시간 축적된 데이터가 전문가의 머릿속에서 패턴화된 결과물인 경우가 많다. 코틀러(Philip Kotler) 역시 마케팅을 \enquote{예술과 과학의 조화}\cite{kotler2021marketing}라고 정의하며, 창의적 직관과 분석적 접근이 모두 필요함을 역설했다.

\textbf{내재화된 패턴}

직관은 단순한 추측이 아니다. 그것은 수많은 성공과 실패의 경험이 뇌에 축적되어 만들어진 고도의 \index{패턴 인식}\enquote{패턴 인식}\sidenote{무질서해 보이는 정보 속에서 의미 있는 규칙이나 구조를 발견하는 인지 과정.}(Pattern Recognition) 능력이다. 숙련된 마케터가 데이터를 하나하나 뜯어보지 않고도 캠페인의 성공 가능성을 감지하는 것은, 그들의 두뇌가 이미 무수한 학습을 통해 정교한 예측 모델을 갖추고 있기 때문이다. 이러한 직관은 의사결정 속도를 높이고, 데이터가 불충분한 상황에서도 방향을 잡게 해주는 강력한 무기다.\sidenote{휴리스틱(Heuristics)이 복잡한 문제를 단순화하기 위한 효율적 \enquote{어림짐작}이라면, 전문가의 패턴 인식은 축적된 경험을 바탕으로 한 정교한 \enquote{\texttt{확률적 판단}}에 가깝다.}

\textbf{체화된 지식의 한계}

그러나 직관에는 치명적인 한계가 있다. 바로 \enquote{공유될 수 없다}는 점이다. 
마이클 폴라니(Michael Polanyi)가 말한 \enquote{체화된 지식(Tacit Knowledge)}\cite{polanyi1966tacit}처럼, 직관은 언어로 설명하기 어려운 개인의 내면에 머무른다. \enquote{왜 그렇게 생각합니까?}라는 질문에 \enquote{그냥 느낌이 그래}라고밖에 답할 수 없다면, 그 지식은 타인에게 전달될 수도, 검증될 수도 없다.

이는 앞서 1장에서 논의했던 신뢰의 조건과 정면으로 배치된다. 우리는 신뢰가 정보의 비대칭성을 해소하고 투명한 맥락을 공유할 때 형성된다고 정의했다. 하지만 마케터가 자신의 직관에만 의존하여 의사결정을 내린다면, 고객과 이해관계자들에게 \enquote{검증 가능한 정보}를 제공할 수 없게 된다. \enquote{전문가인 나를 믿으라}는 말은 더 이상 복잡한 현대 시장에서 통용되지 않는다. 판단의 근거가 불투명한 \enquote{블랙박스} 안에서는 진정한 의미의 신뢰 관계가 싹틀 수 없기 때문이다.

\textbf{구조화의 필요성}

구조화의 핵심 목적은 직관을 대체하는 것이 아니라, 직관이 건너뛴 논리적 연결고리를 복원하는 데 있다. 직관이 설명 불가능했던 이유는 뇌가 수많은 변수를 무의식적으로 빠르게 통합하여 결론을 내리는 \enquote{비선형적}\sidenote{입력(원인)과 출력(결과)의 관계가 단순 비례하지 않고, 수많은 변수가 복잡하게 상호작용하는 상태. 결과만으로는 어떤 변수가 결정적이었는지 역추적하기 어렵다.} 과정이기 때문이다. 이 \enquote{블랙박스}를 역설계하여, 어떤 맥락과 데이터가 그러한 판단을 이끌어냈는지 \enquote{사고의 경로}를 추적하여 재구성해야 한다.

\begin{marginfigure}
    \includegraphics[width=\linewidth]{figures/ch02/non-linear-process-vertical.png}
    \caption{비선형적 직관(블랙박스)과 선형적 논리(투명박스)의 비교}
    \label{fig:nonlinear-vs-linear}
\end{marginfigure}

구조화가 필요한 이유는 명확하다. 설명되지 않은 성공은 재현할 수 없는 \enquote{운}에 불과하며, 원인을 알 수 없는 실패는 개선의 단서를 주지 못하기 때문이다. 모호했던 \enquote{감}을 명확한 \enquote{논리}의 구조로 풀어낼 때, 우리는 비로소 판단의 오류를 찾아내 수정할 수 있고, 타인을 설득할 수 있는 객관적 근거를 확보하게 된다. 이것이 진정한 의미의 \enquote{설명 가능한 마케팅}으로 나아가는 첫걸음이다.
