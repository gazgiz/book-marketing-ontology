%%%%%%%%%%%%%%%%%%%%%%%%%%%%%%%%%%%%%%%%%
% kaobook
% LaTeX Template
% Version 1.3 (December 9, 2021)
%
% This template originates from:
% https://www.LaTeXTemplates.com
%
% For the latest template development version and to make contributions:
% https://github.com/fmarotta/kaobook
%
% Authors:
% Federico Marotta (federicomarotta@mail.com)
% Based on the doctoral thesis of Ken Arroyo Ohori (https://3d.bk.tudelft.nl/ken/en)
% and on the Tufte-LaTeX class.
% Modified for LaTeX Templates by Vel (vel@latextemplates.com)
%
% License:
% CC0 1.0 Universal (see included MANIFEST.md file)
%
%%%%%%%%%%%%%%%%%%%%%%%%%%%%%%%%%%%%%%%%%

%----------------------------------------------------------------------------------------
%	PACKAGES AND OTHER DOCUMENT CONFIGURATIONS
%----------------------------------------------------------------------------------------

\documentclass[
	b5paper, % Page size
	fontsize=10pt, % Base font size
	twoside=true, % Use different layouts for even and odd pages (in particular, if twoside=true, the margin column will be always on the outside)
	%open=any, % If twoside=true, uncomment this to force new chapters to start on any page, not only on right (odd) pages
	%chapterentrydots=true, % Uncomment to output dots from the chapter name to the page number in the table of contents
	numbers=noenddot, % Comment to output dots after chapter numbers; the most common values for this option are: enddot, noenddot and auto (see the KOMAScript documentation for an in-depth explanation)
]{kaobook}

\usepackage[disablepatch=caption]{tocbasic}

\usepackage{fontspec}
\usepackage{xeCJK}
\xeCJKsetup{space=true}

\setmainfont{NotoSerifCJKkr-Regular.otf}[
  Path=fonts/noto/,
  BoldFont=NotoSerifCJKkr-Bold.otf,
  ItalicFont=NotoSerifCJKkr-Regular.otf,
  ItalicFeatures={FakeSlant=0.2},
  BoldItalicFont=NotoSerifCJKkr-Bold.otf,
  BoldItalicFeatures={FakeSlant=0.2}
]
\setCJKmainfont{NotoSerifCJKkr-Regular.otf}[
  Path=fonts/noto/,
  BoldFont=NotoSerifCJKkr-Bold.otf,
  ItalicFont=NotoSerifCJKkr-Regular.otf,
  ItalicFeatures={FakeSlant=0.2},
  BoldItalicFont=NotoSerifCJKkr-Bold.otf,
  BoldItalicFeatures={FakeSlant=0.2}
]
\setCJKsansfont{NotoSansCJKkr-Regular.otf}[
  Path=fonts/noto/,
  BoldFont=NotoSansCJKkr-Bold.otf,
  ItalicFont=NotoSansCJKkr-Regular.otf,
  ItalicFeatures={FakeSlant=0.2},
  BoldItalicFont=NotoSansCJKkr-Bold.otf,
  BoldItalicFeatures={FakeSlant=0.2}
]
\setCJKmonofont{NotoSansCJKkr-Regular.otf}[
  Path=fonts/noto/,
  BoldFont=NotoSansCJKkr-Bold.otf,
  ItalicFont=NotoSansCJKkr-Regular.otf,
  ItalicFeatures={FakeSlant=0.2},
  BoldItalicFont=NotoSansCJKkr-Bold.otf,
  BoldItalicFeatures={FakeSlant=0.2}
]
%     \newfontfamily\hangulfont{NotoSerifCJKkr-Regular.otf}[
%       Path=fonts/noto/,
%       BoldFont=NotoSerifCJKkr-Bold.otf,
%       ItalicFont=NotoSerifCJKkr-Regular.otf,
%       ItalicFeatures={FakeSlant=0.2},
%       BoldItalicFont=NotoSerifCJKkr-Bold.otf,
%       BoldItalicFeatures={FakeSlant=0.2}
%     ]
%     \newfontfamily\hangulfontsf{NotoSansCJKkr-Regular.otf}[
%       Path=fonts/noto/,
%       BoldFont=NotoSansCJKkr-Bold.otf,
%       ItalicFont=NotoSansCJKkr-Regular.otf,
%       ItalicFeatures={FakeSlant=0.2},
%       BoldItalicFont=NotoSansCJKkr-Bold.otf,
%       BoldItalicFeatures={FakeSlant=0.2}
%     ]
%     \newfontfamily\hangulfonttt{NotoSansCJKkr-Regular.otf}[
%       Path=fonts/noto/,
%       BoldFont=NotoSansCJKkr-Bold.otf,
%       ItalicFont=NotoSansCJKkr-Regular.otf,
%       ItalicFeatures={FakeSlant=0.2},
%       BoldItalicFont=NotoSansCJKkr-Bold.otf,
%       BoldItalicFeatures={FakeSlant=0.2}
%     ]    
\usepackage{setspace}
    
\usepackage{tikz}
\usetikzlibrary{arrows.meta, positioning, calc}

% Choose the language
\ifxetexorluatex
% 	\usepackage{polyglossia}
% 	\setmainlanguage{korean}
%     \setotherlanguage{english}
    \usepackage[english]{babel}
\else
	\usepackage[english]{babel} % Load characters and hyphenation
\fi

\usepackage[
  backend=biber,
  style=authoryear,
  language=auto,
  autolang=other
]{biblatex}

\usepackage[english=british]{csquotes}	% English quotes

% Load packages for testing
\usepackage{blindtext}
%\usepackage{showframe} % Uncomment to show boxes around the text area, margin, header and footer
%\usepackage{showlabels} % Uncomment to output the content of \label commands to the document where they are used

% Load the bibliography package
\usepackage{kaobiblio}
\addbibresource{main.bib} % Bibliography file

% Load mathematical packages for theorems and related environments
\usepackage[framed=true]{kaotheorems}

% Load the package for hyperreferences
\usepackage{kaorefs}

\graphicspath{{examples/documentation/images/}{images/}} % Paths in which to look for images

\makeindex[columns=3, title=Alphabetical Index, intoc] % Make LaTeX produce the files required to compile the index

\makeglossaries % Make LaTeX produce the files required to compile the glossary
\input{glossary.tex} % Include the glossary definitions

\makenomenclature % Make LaTeX produce the files required to compile the nomenclature

% Reset sidenote counter at chapters
%\counterwithin*{sidenote}{chapter}

%----------------------------------------------------------------------------------------
\DeclareLanguageMapping{korean}{english}
\DeclareQuoteAlias{british}{korean}

\begin{document}

%----------------------------------------------------------------------------------------
%	BOOK INFORMATION
%----------------------------------------------------------------------------------------

\titlehead{기획자를 위한 \texttt{온톨로지} 접근 가이드}
\subject{Use this document as a template}

%\title[Example and documentation of the {\normalfont\texttt{kaobook}} class]{Example and documentation \\ of the {\normalfont\texttt{kaobook}} class}
\title[마케팅을 이해하는 {\texttt{온톨로지}}]{마케팅을 이해하는 {\texttt{온톨로지}}}
\subtitle{논리로 설계하는 관계의 기술}

\author[김정석]{김정석\thanks{Ontology \& Neuro-Symbolic AI Researcher}}

\date{\today}

\publishers{An Awesome Publisher}

%----------------------------------------------------------------------------------------

\frontmatter % Denotes the start of the pre-document content, uses roman numerals

%----------------------------------------------------------------------------------------
%	OPENING PAGE
%----------------------------------------------------------------------------------------

%\makeatletter
%\extratitle{
%	% In the title page, the title is vspaced by 9.5\baselineskip
%	\vspace*{9\baselineskip}
%	\vspace*{\parskip}
%	\begin{center}
%		% In the title page, \huge is set after the komafont for title
%		\usekomafont{title}\huge\@title
%	\end{center}
%}
%\makeatother

%----------------------------------------------------------------------------------------
%	COPYRIGHT PAGE
%----------------------------------------------------------------------------------------

\makeatletter
\uppertitleback{\@titlehead} % Header

\lowertitleback{
	\textbf{Disclaimer}\\
	You can edit this page to suit your needs. For instance, here we have a no copyright statement, a colophon and some other information. This page is based on the corresponding page of Ken Arroyo Ohori's thesis, with minimal changes.
	
	\medskip
	
	\textbf{No copyright}\\
	\cczero\ This book is released into the public domain using the CC0 code. To the extent possible under law, I waive all copyright and related or neighbouring rights to this work.
	
	To view a copy of the CC0 code, visit: \\\url{http://creativecommons.org/publicdomain/zero/1.0/}
	
	\medskip
	
	\textbf{Colophon} \\
	This document was typeset with the help of \href{https://sourceforge.net/projects/koma-script/}{\KOMAScript} and \href{https://www.latex-project.org/}{\LaTeX} using the \href{https://github.com/fmarotta/kaobook/}{kaobook} class.
	
	The source code of this book is available at:\\\url{https://github.com/fmarotta/kaobook}
	
	(You are welcome to contribute!)
	
	\medskip
	
	\textbf{Publisher} \\
	First printed in May 2019 by \@publishers
}
\makeatother

%----------------------------------------------------------------------------------------
%	DEDICATION
%----------------------------------------------------------------------------------------

\dedication{
%	The harmony of the world is made manifest in Form and Number, and the heart and soul and all the poetry of Natural Philosophy are embodied in the concept of mathematical beauty.\\
%	\flushright -- D'Arcy Wentworth Thompson

\begin{flushright}
스스로의 길을 찾기 위해 질문을 멈추지 않고,\\
때로는 멈춰 서더라도 끝내 앞으로 나아가려는\\
모든 이들에게 이 책을 바칩니다.\\[1.5em]

당신의 걸음은 언제나 의미 있으며,\\
그 여정은 생각보다 더 멀리 닿을 수 있습니다.\\[1.5em]

\end{flushright}
}

%----------------------------------------------------------------------------------------
%	OUTPUT TITLE PAGE AND PREVIOUS
%----------------------------------------------------------------------------------------

% Note that \maketitle outputs the pages before here

\maketitle
 
%----------------------------------------------------------------------------------------
%	PREFACE
%----------------------------------------------------------------------------------------

\chapter*{서문}
\addcontentsline{toc}{chapter}{서문} % Add the preface to the table of contents as a chapter
\setstretch{1.6}
마케팅과 온톨로지는 모두 {\normalfont\texttt{관계}}라는 단어를 사용한다. 그러나 같은 단어를 사용한다고 해서 이들이 같은 의미를 담고 있는 것은 아니다. 두 영역에서 관계라는 단어는 서로 완전히 다른 개념적
틀을 가지고 사용된다. 이 점을 서문의 시작에서 분명하게 하고자 한다.

먼저 \texttt{마케팅에서의 관계(relationship)}는 주로 사람과 사람, 혹은 고객과 브랜드 사이의 심리적이고 정서적인 유대감을 의미한다. 이 관계는 신뢰, 충성도, 애착, 만족감과 같은 감성적이며 주
관적인 연결성을 나타낸다. 마케팅은 바로 이 유대감을 형성하고 유지하며 강화하는 활동으로, 최근의 마케팅 연구들은 이를 \enquote{관계의 설계}\parencite{gummesson2008}로 표현하기도 한다.
반면, \texttt{온톨로지에서의 관계(relation)}는 주관적이거나 감성적이지 않다. 온톨로지에서의 관계는 명확히 정의된 \texttt{엔티티(entity)}들 사이의 구조적이고 객관적인 연결이다. 특정 \texttt{주체(subject)}와 \texttt{객체(object)}가 어떤 \texttt{의미적 관계(predicate)}로 연결되어 있는지 논리적으로 명시하는 것이 온톨로지적 관계이다.
예를 들어,\enquote{고객이 제품을 구매하다}, 혹은 \enquote{사용자가 서비스를 이용하다}와 같은 관계가 온톨로지적 의미의 관계이다.

이 책은 마케팅을 온톨로지적 방법으로 분석하고 활용하는 것을 목적으로 한다. 그렇기에 두 영역에서 사용되는 용어와 개념을 혼동하지 않도록 초반부터 명확히 구분하는 것이 중요하다. 마케팅적 관계
의 감성적이고 심리적인 요소는, 온톨로지에서 사용되는 객관적이고 구조적인 관계로 명료하게 재구성되어야 한다. 이를 통해 비로소 마케팅을 온톨로지적으로 명확하게 분석하고, 설계하고, 실행할 수 있다.

\begin{flushright}
	\textit{저자}
\end{flushright}

\setstretch{1.2}
\index{preface}

%----------------------------------------------------------------------------------------
%	TABLE OF CONTENTS & LIST OF FIGURES/TABLES
%----------------------------------------------------------------------------------------

\begingroup % Local scope for the following commands

% Define the style for the TOC, LOF, and LOT
%\setstretch{1} % Uncomment to modify line spacing in the ToC
%\hypersetup{linkcolor=blue} % Uncomment to set the colour of links in the ToC
\setlength{\textheight}{230\hscale} % Manually adjust the height of the ToC pages

% Turn on compatibility mode for the etoc package
\etocclasstocstyle % "toc display" as if etoc was not loaded
\etocstandardlines % "toc lines" as if etoc was not loaded

\tableofcontents % Output the table of contents

\listoffigures % Output the list of figures

% Comment both of the following lines to have the LOF and the LOT on different pages
\let\cleardoublepage\bigskip
\let\clearpage\bigskip

\listoftables % Output the list of tables

\endgroup

%----------------------------------------------------------------------------------------
%	MAIN BODY
%----------------------------------------------------------------------------------------

\mainmatter % Denotes the start of the main document content, resets page numbering and uses arabic numbers
\setchapterstyle{kao} % Choose the default chapter heading style

\pagelayout{wide} % No margins
\addpart{마케팅, 판단의 기술}
%\pagelayout{wide} % No margins

\pagelayout{margin}
\setstretch{1.8}
\setchapterpreamble[u]{\margintoc}
\chapter{메시지 과잉의 시대}
\labch{messageoverflow}

\input{chapters/01/0001.tex}
\section{관계의 목적}
앞에서 우리는 고객, 브랜드, 제품, 채널, 시간, 그리고 고객의 내면 요소까지 포함한 관계의 구조를 살펴보았다. 그러나 관계의 언어를 아무리 정교하게 설명하더라도, 한 가지 질문이 남는다. 이 모든 관계를 왜 설계하려 하는가, 그리고 그 관계는 무엇을 위해 존재해야 하는가 하는 문제다. 관계의 목적이 불명확한 상태에서 이루어지는 마케팅 활동은, 결국 더 많은 메시지를 더 자주 보내자는 수준의 결론으로 흘러가기 쉽다. 메시지 과잉의 시대에는 이것이 곧 노이즈의 생산을 의미한다.

관계의 목적을 이야기할 때, 가장 먼저 사라지기 쉬운 것이 ‘제품’이다. 고객과 브랜드의 심리적 유대만 강조하다 보면, 정작 무엇을 매개로 관계가 형성되고 유지되는지에 대한 논의가 뒤로 밀린다.
하지만 고객은 브랜드 자체를 소비하지 않는다. 고객은 자신의 문제 상황을 해결해 줄 수단으로서 제품을 구매하고, 그 제품이 만들어 내는 경험을 통해 브랜드를 판단한다.
즉, 제품은 고객–브랜드 관계가 현실에서 구현되는 매개체이자, 관계의 목적이 구체적인 형태로 드러나는 자리다. 따라서 관계의 목적을 이해하려면, 결국 \enquote{고객이 어떤 제품을 어떻게 선택하도록 돕고 있는가}라는 질문으로 돌아가야 한다.
관계는 단지 \enquote{좋은 사이}를 유지하기 위한 감정적 장치가 아니다. 마케팅 관점에서 관계의 가장 근본적인 목적은 바로 이 선택 과정에서의 판단 비용을 줄이는 데 있다.
고객이 매번 새로운 선택을 할 때마다 모든 제품을 처음부터 비교·분석할 수는 없다. 이때 고객은 브랜드와 관계뿐 아니라, 특정 제품군과의 누적된 경험을 함께 사용한다. 
\enquote{이 브랜드의 이 라인 제품은 대략 이런 품질과 이런 느낌을 줄 것이다}라는 기대가 형성되어 있을 때, 고객은 복잡한 계산 없이도 특정 브랜드–제품 묶음을 빠르게 선택할 수 있다. 
관계는 이렇게 고객의 머릿속에 \enquote{브랜드–제품 패턴}을 만들어 놓고, 그 패턴을 통해 판단의 지름길(\index{휴리스틱}휴리스틱\sidenote{휴리스틱은 심리학,인지과학에서 자주쓰이는 말로 모든 정보를 완벽하게 따져보지 못할때, 경험과 직관을 활용해 \enquote{그럴듯한} 결론에 빠르게 도달하게 해주는 판단의 요령을 뜻한다.})을 제공한다.

브랜드 입장에서 보았을 때도 마찬가지다. 브랜드는 시장에 존재하는 모든 잠재 고객을 동일한 강도로 설득할 수 없다. 누구에게, 어떤 강도로 다가갈 것인지 선택해야 한다. 이때 관계는 한정된 자원을 어디에 우선 배분할 것인지 결정하게 해 주는 기준이 된다. 이미 일정 수준의 신뢰와 상호 이해가 형성된 고객은 그렇지 않은 고객에 비해 더 낮은 설득 비용으로 더 높은 반응을 기대할 수 있는 집단이 된다. 관계의 목적은 따라서, 고객과 브랜드 양쪽 모두의 판단 비용을 동시에 낮추는 데 있다.


\section{신뢰의 조건}

\section{투명성의 기반}

\setchapterpreamble[u]{\margintoc}
\chapter{판단의 기준}
\labch{criteriaofjudgement}
\section{직관의 한계}
\section{맥락의 파편화}
\section{구조적 해결책}

\pagelayout{wide} % No margins
\addpart{온톨로지, 논리의 설계}
\pagelayout{margin} % No margins

\setchapterpreamble[u]{\margintoc}
\chapter{온톨로지의 본질}
\labch{originofontology}
\section{의미의 합의}
\section{관계의 구조화}
\section{논리의 검증}

\setchapterpreamble[u]{\margintoc}
\chapter{모델링의 실제}
\labch{practicalmodeling}
\section{욕구의 계층화}
\section{행동의 인과성}
\section{시간의 축적}

\pagelayout{wide} % No margins
\addpart{경험, 구조의 확장}
\pagelayout{margin} % No margins

\setchapterpreamble[u]{\margintoc}
\chapter{일관된 접점의 구현}
\labch{consistenttouchpoint}
\section{데이터의 통합}
\section{여정의 시각화}
\section{맥락의 최적화}

\setchapterpreamble[u]{\margintoc}
\chapter{관계 설계자의 미래}
\labch{futureofontologist}
\section{인공지능과 협업}
\section{초개인화의 완성}
\section{설계자의 태도}

%\input{chapters/introduction.tex}

%\pagelayout{wide} % No margins
%\addpart{Class Options, Commands and Environments}
%\pagelayout{margin} % Restore margins

%\input{chapters/options.tex}
%\input{chapters/textnotes.tex}
%\input{chapters/figsntabs.tex}
%\input{chapters/references.tex}

%\pagelayout{wide} % No margins
%\addpart{Design and Additional Features}
%\pagelayout{margin} % Restore margins


%\input{chapters/layout.tex}
%\input{chapters/mathematics.tex}

\appendix % From here onwards, chapters are numbered with letters, as is the appendix convention

\pagelayout{wide} % No margins
\addpart{Appendix}
\pagelayout{margin} % Restore margins

%\input{chapters/appendix.tex}

%----------------------------------------------------------------------------------------

\backmatter % Denotes the end of the main document content
\setchapterstyle{plain} % Output plain chapters from this point onwards

%----------------------------------------------------------------------------------------
%	BIBLIOGRAPHY
%----------------------------------------------------------------------------------------

% The bibliography needs to be compiled with biber using your LaTeX editor, or on the command line with 'biber main' from the template directory

\defbibnote{bibnote}{Here are the references in citation order.\par\bigskip} % Prepend this text to the bibliography
\printbibliography[heading=bibintoc, title=Bibliography, prenote=bibnote] % Add the bibliography heading to the ToC, set the title of the bibliography and output the bibliography note

%----------------------------------------------------------------------------------------
%	NOMENCLATURE
%----------------------------------------------------------------------------------------

% The nomenclature needs to be compiled on the command line with 'makeindex main.nlo -s nomencl.ist -o main.nls' from the template directory

\nomenclature{$c$}{Speed of light in a vacuum inertial frame}
\nomenclature{$h$}{Planck constant}

\renewcommand{\nomname}{Notation} % Rename the default 'Nomenclature'
\renewcommand{\nompreamble}{The next list describes several symbols that will be later used within the body of the document.} % Prepend this text to the nomenclature

\printnomenclature % Output the nomenclature

%----------------------------------------------------------------------------------------
%	GREEK ALPHABET
% 	Originally from https://gitlab.com/jim.hefferon/linear-algebra
%----------------------------------------------------------------------------------------

\vspace{1cm}

{\usekomafont{chapter}Greek Letters with Pronunciations} \\[2ex]
\begin{center}
	\newcommand{\pronounced}[1]{\hspace*{.2em}\small\textit{#1}}
	\begin{tabular}{l l @{\hspace*{3em}} l l}
		\toprule
		Character & Name & Character & Name \\ 
		\midrule
		$\alpha$ & alpha \pronounced{AL-fuh} & $\nu$ & nu \pronounced{NEW} \\
		$\beta$ & beta \pronounced{BAY-tuh} & $\xi$, $\Xi$ & xi \pronounced{KSIGH} \\ 
		$\gamma$, $\Gamma$ & gamma \pronounced{GAM-muh} & o & omicron \pronounced{OM-uh-CRON} \\
		$\delta$, $\Delta$ & delta \pronounced{DEL-tuh} & $\pi$, $\Pi$ & pi \pronounced{PIE} \\
		$\epsilon$ & epsilon \pronounced{EP-suh-lon} & $\rho$ & rho \pronounced{ROW} \\
		$\zeta$ & zeta \pronounced{ZAY-tuh} & $\sigma$, $\Sigma$ & sigma \pronounced{SIG-muh} \\
		$\eta$ & eta \pronounced{AY-tuh} & $\tau$ & tau \pronounced{TOW (as in cow)} \\
		$\theta$, $\Theta$ & theta \pronounced{THAY-tuh} & $\upsilon$, $\Upsilon$ & upsilon \pronounced{OOP-suh-LON} \\
		$\iota$ & iota \pronounced{eye-OH-tuh} & $\phi$, $\Phi$ & phi \pronounced{FEE, or FI (as in hi)} \\
		$\kappa$ & kappa \pronounced{KAP-uh} & $\chi$ & chi \pronounced{KI (as in hi)} \\
		$\lambda$, $\Lambda$ & lambda \pronounced{LAM-duh} & $\psi$, $\Psi$ & psi \pronounced{SIGH, or PSIGH} \\
		$\mu$ & mu \pronounced{MEW} & $\omega$, $\Omega$ & omega \pronounced{oh-MAY-guh} \\
		\bottomrule
	\end{tabular} \\[1.5ex]
	Capitals shown are the ones that differ from Roman capitals.
\end{center}

%----------------------------------------------------------------------------------------
%	GLOSSARY
%----------------------------------------------------------------------------------------

% The glossary needs to be compiled on the command line with 'makeglossaries main' from the template directory

\setglossarystyle{listgroup} % Set the style of the glossary (see https://en.wikibooks.org/wiki/LaTeX/Glossary for a reference)
\printglossary[title=Special Terms, toctitle=List of Terms] % Output the glossary, 'title' is the chapter heading for the glossary, toctitle is the table of contents heading

%----------------------------------------------------------------------------------------
%	INDEX
%----------------------------------------------------------------------------------------

% The index needs to be compiled on the command line with 'makeindex main' from the template directory

\printindex % Output the index

%----------------------------------------------------------------------------------------
%	BACK COVER
%----------------------------------------------------------------------------------------

% If you have a PDF/image file that you want to use as a back cover, uncomment the following lines

%\clearpage
%\thispagestyle{empty}
%\null%
%\clearpage
%\includepdf{cover-back.pdf}

%----------------------------------------------------------------------------------------

\end{document}
